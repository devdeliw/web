\documentclass[svgnames]{article}     % use "amsart" instead of "article" for AMSLaTeX format
%\geometry{landscape}                 % Activate for rotated page geometry

%\usepackage[parfill]{parskip}        % Activate to begin paragraphs with an empty line rather than an indent

\usepackage{graphicx}                 % Use pdf, png, jpg, or eps§ with pdflatex; use eps in DVI mode

%maths                                % TeX will automatically convert eps --> pdf in pdflatex
\usepackage{amssymb}
\usepackage{amsmath}
\usepackage{esint}
\usepackage{geometry}

% Inverting Color of PDF
%\usepackage{xcolor}
%\pagecolor[rgb]{0.19,0.19,0.19}
%\color[rgb]{0.77,0.77,0.77}

%noindent
\setlength\parindent{0pt}

%pgfplots
\usepackage{pgfplots}

%images
\graphicspath{{/Users/devaldeliwala/screenshots/}}                   % Activate to set a image directory

%tikz
\usepackage{pgfplots}
\pgfplotsset{compat=1.15}
\usepackage{comment}
\usetikzlibrary{arrows}
\usepackage[most]{tcolorbox}

%Figures
\usepackage{float}
\usepackage{caption}
\usepackage{lipsum}


\title{Lab 0 Report}
\author{Deval Deliwala}
%\date{}                              % Activate to display a given date or no date

\begin{document}
\maketitle
%\section{}
%\subsection{}
\tableofcontents % Activate to display a table of contents

\section{Exercise - Parallax}

\textbf{Setup}: Take two pencils.  Hold one of them in front of you at arm’s
length and the other at roughly half that distance.  Close one eye and focus on
a point far away. Then move your head side to side and record what you observe \\

\textbf{Observation}: When moving my head side-to-side, the further pencil
seems to be near-stationary while the closer pencil moves a lot more relative
to my eyes. When the pencils are \textit{the same distance} away from my eye,
both appear to move equally when my head oscillates. \\ 

\textbf{Experimental Design}: The distance is inversely proportional to how
much the pencils seem to move relative to my eyes. If the pencil moves less, in
this case the further one, it is farther away. Therefore, the formula for how
far an object is should be of the form $d = 1 / \theta$ \\

\section{Experiment 1 - The Law of Reflection}

The goal of this experiment is to demonstrate the law of reflection by
comparing the incident and reflected angles of a laser incident on a plane
mirror. \\

We place the plane mirror on the top of the protractor where the mirror is
normal to the 0 degree axis. To ensure the mirror is calibrated such that the
incident and reflected angles should equate, we first shot a laser at the 45
degree mark as it has a much longer tick mark which helps with aiming the
laser. We rotated and aligned the plane mirror until the 45 degree incident
also resulted in a 45 degree reflection. \\

Of course, the goal of this experiment itself is to demonstrate this equality,
but it is intuitively obvious and so we used it to help in our initial
calibration\\ 

\subsection{Experimental Design} 

\begin{figure}[H]
  \centering
    \includegraphics[width = 10cm]{screenshot 18.png}
    \caption{Experiment 1 Setup}
\end{figure}

\paragraph{Verify / Refine:} \mbox{} \\

To check whether our lab setup is viable we checked our precision errors. After
initially setting up the plane mirror such that the incident and reflected
angles are equal, we performed some more checks on angles that did not have as
large of a guideline as the 45 degree angle. \\

For 30 degrees, our reflected angle was half a degree smaller (29.5) which we
assume means our mirror isn't perfectly straight, or the width of the laser
itself causes some uncertainty. After the light turned off, our measurements
became much better. 

\paragraph{Uncertainties:} \mbox{} \\

\begin{itemize}
  \item[-] Width of the laser -- does not change over the angle range (minimize
    by looking at center of beam) 
    \item[-] Mirror angle/translation offsets - minimize by calibrating
      beforehand 
    \item[-] Precision error of the protractor - constant for the whole
      experiment 
    \item[-] Error in the angle of the laser emitter (minimized by ensuring the
      reflection is off the center of the protractor) - changes based on each
      measurement
    \item[-] Turning the lights off makes the beam much more clear
    \item[-] Line up the laser such that the incident occurs at the center of
      0 degree axis -- helps especially for smaller angles 
\end{itemize}


\subsection{Data Collection}
\mbox{}  
\begin{center}
\begin{tabular}{ |c|c|c|c| }
 \hline
 Incident & Deval  & Michaela & SuryaNeil\\ 
 \hline
 5 & 5.5 & 5.1 & 5.05 \\ 
 \hline
 20 & 20.3 & 20.1 & 20.5 \\
 \hline
 40 & 40.5 & 41 & 40 \\
 \hline
 60 & 60 & 60 & 60 \\
 \hline
 80 & 80 & 80 & 80 \\
 \hline
\end{tabular}
\end{center}

The precision error for the protractor is 0.5 degrees. We began with 5 degrees
as 0 degrees just illuminated the edge of the mirror without a reflection. We
found that for larger angles, we got more and more precise. This is likely
because, from the perspective of larger angles, it is easier to line up the
incident to occur exactly at 0 degrees at the center of the mirror. This will
yield more accurate reflected angles which in turn yields a better accuracy and
precision. 


\begin{figure}[H]
  \centering
    \includegraphics[width = 15cm]{screenshot 22.png}
    \caption{Data Collection for Experiment 1}
\end{figure}

They may not be visible, but the above graph does include error bars. For both
horizontal and vertical we set an above and below error of 0.25 to account for
0.5 degrees error total. Since all of the error bars do intersect the best-fit
  curve, we can consider the linear regression a good fit to the data. \\

Including the law of reflection itself in the graph -- $\theta_{in}
= \theta_{out}$ yields the following graph: 

\begin{figure}[H]
  \centering
    \includegraphics[width = 15cm]{screenshot 23.png}

    \caption{Data Collection vs. Law of Reflection}
\end{figure}

As you can see our collected data lines up near-perfectly with the law of
reflection. \\

\subsection{Determining Reduced Chi Squared Value}

Implementing the chi square formula: 

\[
\chi^2 = \sum \frac{(y_i - y(x_i))^2}{\alpha_i^2}
\] \vspace{5px}

such that the reduced chi squared $\tilde{\chi^2} = \chi^2 / \nu$, where $\nu
= 15 - 1 = 14$, we get

\begin{align*}
  \chi^2 &= 7.45 \\
  \tilde{\chi^2} &= 0.53
\end{align*}

And since $\tilde{\chi^2} < 1$, we say the model is falling within the expected
uncertainty range. 






\end{document}

